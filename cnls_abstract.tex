\documentclass[letter,nofootinbib,superscriptaddress,twocolumn]{revtex4-1}

\newcommand{\nubhlight}{$\nu\texttt{bhlight}$}

\begin{document}

\title{Black Holes, Neutrinos, Neutrons, et al.: How The Merger of Two
  Dead Stars Make the Heaviest Elements in the Universe, and How We
  Know}

\author{Jonah M. Miller}
\email{jonahm@lanl.gov}
\affiliation{Center for Nonlinear Studies, Los Alamos National Laboratory, Los Alamos, NM, USA}
\affiliation{Center for Theoretical Astrophysics, Los Alamos National Laboratory, Los Alamos, NM, USA}
\affiliation{CCS-2, Los Alamos National Laboratory, Los Alamos, NM, USA}

\begin{abstract}

The 2017 detection of the in-spiral and merger of two neutron stars
was a landmark discovery in astrophysics. We now know that the merger
of these ultracompact stellar remnants is a central engine of short
gamma ray bursts and a site of r-process nucleosynthesis, where the
heaviest elements in our universe are formed. In the coming years, we
expect many more such mergers.

The modeling of these systems depends sensitively on a complex
interplay of general relativity, plasma physics, nuclear physics, and
neutrino physics. As observations ramp up in the near future, detailed
models are urgently needed. This presents a significant computational
challenge along with the observational one.

In this talk, I describe what we saw on September 17, 2017, present
open questions, and present my own contribution to the modeling
effort. This is \nubhlight, a state-of-the-art numerical scheme for
solving general relativistic magnetohydrodynamics with
frequency-dependent neutrino transport using a Monte Carlo method. I
will also show exciting preliminary results from early \nubhlight${}$
simulations.

\end{abstract}

\maketitle

\end{document}