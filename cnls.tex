
%\pdfminorversion=4
\documentclass[]{beamer}
\mode<presentation>
% Time-stamp: <2019-02-06 16:28:17 (jonahm)>

% beamer stuff
% Gives us the bottom line with all the goodies
\useoutertheme{infolines}
% Just the theme to use. Should be built into bemaer. Setting the
% height gets rid of a whole lot of whitespace
\usetheme[height=7mm]{Rochester}
\usefonttheme{serif}
% Usually beamer gives you navigation hyperlinks on the bottom
% right. I turned this off. It's annoying.
\setbeamertemplate{navigation symbols}{} 
% Makes my text boxes look pretty
\setbeamertemplate{blocks}[rounded][shadow=true] 
% Makes my bullet points 3d balls
\setbeamertemplate{items}[ball]

% Josh's packages
\usepackage{multimedia}
\usepackage{graphicx}

% Packages for me
\usepackage{amsmath,amssymb,latexsym,amsthm}
\usepackage[mathscr]{eucal}
\usepackage{mathrsfs}
\usepackage{verbatim}
\usepackage{braket}
\usepackage{listings}
\usepackage{xcolor}
% \usepackage[usenames,dvipsnames,svgnames,table]{xcolor}
\usepackage{fancybox}
\usepackage{animate}
% \usepackage{media9}
\usepackage{multicol}
\usepackage{mdframed}
\usepackage{hyperref}
%\usepackage{scalerel}
\usepackage[outline]{contour}
\contourlength{1.2pt}

% Macros

%Blackboard Bold
\newcommand{\R}{\mathbb{R}}
\newcommand{\Z}{\mathbb{Z}}
\newcommand{\N}{\mathbb{N}}
\newcommand{\Q}{\mathbb{Q}}
\newcommand{\A}{\mathbb{A}}
\newcommand{\E}{\mathbb{E}}
% other
\newcommand{\eval}{\biggr\rvert} %evaluated at
\newcommand{\myvec}[1]{\mathbf{#1}} % vectors for me
% total derivatives 
\newcommand{\diff}[2]{\frac{d #1}{d #2}} 
\newcommand{\dd}[1]{\frac{d}{d #1}}
% partial derivatives
\newcommand{\pd}[2]{\frac{\partial #1}{\partial #2}} 
\newcommand{\pdd}[1]{\frac{\partial}{\partial #1}} 
% Order operator
\DeclareRobustCommand{\orderof}{\ensuremath{\mathcal{O}}}

% braces
\newcommand{\paren}[1]{\left( #1 \right)}
\newcommand{\sqrbrace}[1]{\left[ #1 \right]}
\newcommand{\curlybrace}[1]{\left\{ #1 \right\}}
\newcommand{\inner}[1]{\paren{#1}}
\newcommand{\norm}[1]{\left| #1 \right|_2}

% g
\newcommand{\detg}{\sqrt{-g}}

% neutrinos
\newcommand{\eepsilon}{\epsilon} % energy
\newcommand{\fin}{f\in \{\nu_e,\nu_{\bar{e}},\nu_x\}}
\newcommand{\sign}{\text{sign}(f)}
\newcommand{\jnuf}{j_{\eepsilon,f}}
\newcommand{\etanuf}{\eta_{\eepsilon,f}}
\newcommand{\Inuf}{I_{\eepsilon,f}}
\newcommand{\chinuf}{\chi_{\eepsilon,f}}
\newcommand{\sigmanuf}{\sigma_{\eepsilon,f}}
\newcommand{\alphanuf}{\alpha_{\eepsilon,f}}
\newcommand{\numin}{\nu_{\text{min}}}
\newcommand{\numax}{\nu_{\text{max}}}


% tikz
\usepackage{tikz}
\usetikzlibrary{arrows}
\usetikzlibrary{decorations.pathmorphing}
\usepackage{pgfplots}


% Keys to support piece-wise uncovering of elements in TikZ pictures:
% \node[visible on=<2->](foo){Foo}
% \node[visible on=<{2,4}>](bar){Bar}   % put braces around comma expressions
% 
% Internally works by setting opacity=0 when invisible, which has the 
% adavantage (compared to \node<2->(foo){Foo} that the node is always there, hence
% always consumes space plus that coordinate (foo) is always available.
% 
% The actual command that implements the invisibility can be overriden
% by altering the style invisible. For instance \tikzsset{invisible/.style={opacity=0.2}}
% would dim the "invisible" parts. Alternatively, the color might be set to white, if the
% output driver does not support transparencies (e.g., PS) 
% 
\tikzset{
  invisible/.style={opacity=0},
  visible on/.style={alt={#1{}{invisible}}},
  alt/.code args={<#1>#2#3}{%
    \alt<#1>{\pgfkeysalso{#2}}{\pgfkeysalso{#3}} % \pgfkeysalso doesn't change the path
  },
}

% some nice flowchart features
\tikzset{
    mynode/.style={rectangle,rounded corners,draw=black, top color=white, bottom color=yellow!50,very thick, inner sep=1em, minimum size=3em, text centered},
    myarrow/.style={->, >=latex', shorten >=1pt, thick},
    mylabel/.style={text width=7em, text centered} 
}  

% squigly arrow
\tikzset{zigzag it/.style={decorate, decoration=zigzag}}

\newcommand{\PI}{3.14}

% define a really nice visible "purple"
\definecolor{gimppurple}{HTML}{AD26FB}
% a light grey
\definecolor{lightgrey}{HTML}{E0E0E0}
% for highlighting
\definecolor{deepblue}{rgb}{0,0,0.5}
\definecolor{deepred}{rgb}{0.6,0,0}
\definecolor{deepgreen}{rgb}{0,0.5,0}

% fonts
% Default fixed font does not support bold face
\DeclareFixedFont{\ttb}{T1}{txtt}{bx}{n}{12} % for bold
\DeclareFixedFont{\ttm}{T1}{txtt}{m}{n}{12}  % for normal

% Python style for highlighting
\newcommand\pythonstyle{\lstset{
language=Python,
basicstyle=\ttm,
otherkeywords={self},
keywordstyle=\ttb\color{deepblue},
emph={__init__},           
emphstyle=\ttb\color{deepred},
commentstyle=\ttfamily\color{deepred},
stringstyle=\color{deepgreen},
frame=tb,                     
showstringspaces=false        
}}

% Python environment
\lstnewenvironment{python}[1][]
{
\pythonstyle
\lstset{#1}
}
{}

\newcommand{\backupbegin}{
   \newcounter{finalframe}
   \setcounter{finalframe}{\value{framenumber}}
}
\newcommand{\backupend}{
   \setcounter{framenumber}{\value{finalframe}}
}

% Automatically generates section breaker slides
\AtBeginSection[]{
  \begin{frame}[plain]
  \vfill
  \centering
  \begin{beamercolorbox}[sep=8pt,center,shadow=true,rounded=true]{title}
    \usebeamerfont{title}\insertsectionhead\par%
  \end{beamercolorbox}
  \vfill
  \end{frame}
}

\title[CNLS seminar]{Black Holes, Neutrinos, Neutrons, et al.}
\subtitle{How The Merger of Two Dead Stars Makes the Heaviest Elements
  in the Universe, and How We Know}
\author[J. Miller]{\textcolor{blue}{Jonah M. Miller},\\B. Ryan, J. Dolence, C. Fryer, A. Burrows}
\institute[LANL CTA/CNLS]{LANL CCS-2\\Center for Nonlinear Studies\\Center for Theoretical Astrophysics}

\date[07/02/19]{CNLS Postdoc Seminar}

\graphicspath{{figures/}}

\begin{document}

\begin{frame}[plain]
\titlepage
\end{frame}

\begin{frame}
  \frametitle{Cosmic Gold}
  \begin{center}
    \includegraphics[width=0.9\textwidth]{quantagold_19201}
  \end{center}
  Ashley Mackenzie for Quanta Magazine, March 23, 2017
\end{frame}

\begin{frame}
  \frametitle{Observations Galore!}
  \begin{center}
    \includegraphics[height=7cm]{abbot-timeline}
  \end{center}
  Abbot+, 2017
\end{frame}

\begin{frame}
  \frametitle{Neutron Stars}
  \begin{center}
    \includegraphics[width=0.9\textwidth]{ns-manhattan}
  \end{center}
  Wikimedia Commons
\end{frame}

% \begin{frame}
%   \frametitle{GW170817}
%   \begin{center}
%     \href{https://youtu.be/e7LcmWiclOs}{\includegraphics[height=6cm]{GW170817-rendition}}
%   \end{center}
%   \url{https://youtu.be/e7LcmWiclOs}\\
%   NSF/LIGO/Sonoma State University/A. Simonnet\\
%    NASA's Goddard Space Flight Center/CI Lab
% \end{frame}

\begin{frame}
  \frametitle{Merger Dynamics in Short}
  \begin{center}
    %\includegraphics[height=0.8\textheight]{frames/betabin_000}
    \animategraphics[height=0.8\textheight,every=10,autoplay,loop,controls]
    {5}{frames/betabin_}{000}{374}
  \end{center}
    \textbf{JMM}+, 2016
\end{frame}

\begin{frame}
  \frametitle{Disk Drives a Jet}
  \begin{center}
    \includegraphics[width=0.9\textwidth]{ruiz-2016}
  \end{center}
  Ruiz+, 2016
\end{frame}

\begin{frame}
  \frametitle{The Radioactive Cloud}
  \begin{center}
    \includegraphics[height=7cm,clip,trim={4cm 1.5cm 4cm 0cm}]
      {ejecta-morphology-z-projection}
  \end{center}
  \textbf{JMM}+, 2016
\end{frame}

\begin{frame}
  \frametitle{The r-process}
  \begin{center}
    \includegraphics[width=0.9\textwidth]{skynet_ye_0p13/frame_0001}
  \end{center}
  Courtesy of J. Lippuner (former LANL CNLS fellow)
\end{frame}

\begin{frame}
  \frametitle{The r-process}
  \begin{center}
    %\includegraphics[width=0.9\textwidth]{skynet_ye_0p13/frame_0001}
    \animategraphics[width=0.9\textwidth,every=3,autoplay,loop,controls]
    {3}{skynet_ye_0p13/frame_}{0001}{0108}
  \end{center}
  Courtesy of J. Lippuner (former LANL CNLS fellow)
\end{frame}

\begin{frame}
  \frametitle{The Kilonova}
  \begin{center}
    \includegraphics[width=10cm]{swope-image}
  \end{center}
  M2H/UC Santa Cruz and Carnegie Observatories/Ryan Foley
\end{frame}

\begin{frame}
  \frametitle{Opacity}
  %\setlength{\unitlength}{1cm}
  \resizebox{12cm}{!}{
    \begin{tikzpicture}
      \node[inner sep=0pt] (rp) at (0,0)
      {\includegraphics[width=10cm]{skynet_ye_0p13/frame_0108}};
      \draw[ultra thick,red,<-] (1.85,-2) -- (2,-2.85)
      -- (2.25, -2.85) node[right] {\tiny Opaque to visible light};
      \draw[ultra thick,red,<-] (0.85,-2) -- (0.5,-2.85)
      -- (0.25,-2.85) node[left] {\tiny Not opaque};
    \end{tikzpicture}
  }
\end{frame}

% \begin{frame}
%  \frametitle{Infrared Light Curves}
%  \begin{center}
%    \includegraphics[height=7cm]{tanvir_light_curve}
%  \end{center}
%  Tanvir et al., 2017/HST
% \end{frame}
% 
% \begin{frame}
%  \frametitle{Infrared Light Curves}
%  \begin{center}
%    \includegraphics[width=0.99\textwidth]{tanvir_light_curve_model}
%  \end{center}
%  Tanvir et al., 2017/HST
% \end{frame}

% \begin{frame}
%   \frametitle{Blue Kilonova Comes from the Disk?}
%   \begin{center}
%     \includegraphics[width=7cm]{disk_3d_render}
%   \end{center}
%   \textbf{JMM}+, in prep.
% \end{frame}

\begin{frame}
  \frametitle{Blue Kilonova Comes from the Disk?}
  \begin{center}
    \resizebox{12.cm}{!}{
      \begin{tikzpicture}
        \filldraw[black] (0,0) rectangle ++(12,8);
        \node[right,align=left] at (0,0.25){\color{red}\textbf{JMM}+, in prep.};
        \node[inner sep=0pt] (wind) at (3,4)
        {\includegraphics[width=0.5\textwidth]{wind-3d-render}};
        \node[inner sep=0pt,rectangle,blue,ultra thick] (disk) at (9,5)
        {\includegraphics[width=0.4\textwidth]{disk_image_no_text}};
        \draw[blue,ultra thick] (6.5,3.5) rectangle ++(5,3.);
        \draw[blue,ultra thick] (3,4) -- (6.5,6.5);
        \draw[blue,ultra thick] (3,4) -- (6.5,3.5);
      \end{tikzpicture}
      %\includegraphics[width=7cm]{disk_3d_render}
      }
  \end{center}
\end{frame}


\begin{frame}
  \frametitle{Ingredients In Kilonova Disk Modeling}
  \begin{itemize}
  \item General relativity
    \begin{itemize}
    \item Rotating black hole spacetime
    \end{itemize}
  \item Plasma physics
    \begin{itemize}
    \item Ideal magnetohydrodynamics
    \end{itemize}
  \item Nuclear physics
    \begin{itemize}
    \item Hot gas treated as being in nuclear-statistical equilibrium via \textbf{equation of state}
    \item Cooling outflow treated in postprocessing via \textbf{nuclear reaction networks}
    \end{itemize}
  \item Radiation physics
    \begin{itemize}
    \item Material is opaque to photons, can be incorporated in plasma physics
    \item Material \textit{not} opaque to \textbf{neutrinos}.
    \item Neutrinos can \textit{change the composition of the
        material} by converting neutrons to protons and vice versa.
    \end{itemize}
  \end{itemize}
\end{frame}

\begin{frame}
  \frametitle{Neutrino Transport Matters!}
  \begin{center}
    %\includegraphics[height=0.8\textheight]{leptoneq/frame_0001}
    \animategraphics[height=0.8\textheight,every=5,autoplay,loop,controls]
    {5}{leptoneq/frame_}{0001}{0101}
  \end{center}
  \textbf{JMM}+, in prep.
\end{frame}

\begin{frame}
  \frametitle{Presenting $\nu\texttt{bhlight}$!}
  \begin{itemize}
  \item General relativistic radiation magnetohydrodynamics for kilonova disks
  \item \textbf{Magnetized gas} via \textit{finite volume methods}
    \begin{itemize}
    \item Standard second-order Gudonov scheme
    \item Cell-centered constrained transport for magnetic fields
    \item WENO5 reconstruction
    \item Local Lax-Friedrichs Riemann solver
    \end{itemize}
  \item \textbf{Neutrinos} via \textit{Monte Carlo methods}
    \begin{itemize}
    \item Explicit integration along geodesics
    \item Probabilistic emissivity, absorption, and scattering
    \item Novel biasing scheme ensures all processes well-sampled
    \end{itemize}
  \item \textbf{Coupled} via \textit{operator splitting}
  \item Built on a long history. See, e.g., Gammie et al. 2003,
    Dolence et al. 2009, Ryan et al. 2015.
  \end{itemize}
\end{frame}

\begin{frame}
  \frametitle{The Fundamental Equations}
  \begin{itemize}
  \item Mass conservation:
    \begin{small}
      \begin{displaymath}
        \partial_t \paren{{\color{red}\detg}\rho_0 u^t}
        + \partial_i\paren{{\color{red}\detg}\rho_0u^i} = 0
      \end{displaymath}
    \end{small}
  \item Momentum and Internal Energy Conservation:
    \begin{small}
      \begin{displaymath}
        \partial_t\sqrbrace{{\color{red}\detg} \paren{T^t_{\ \nu} + \rho_0u^t \delta^t_\nu}}
        + \partial_i\sqrbrace{{\color{red}\detg}\paren{T^i_{\ \nu} + \rho_0 u^i \delta^t_\nu}}
        = {\color{red}\detg} \paren{T^\kappa_{\ \lambda} {\color{red}\Gamma^\lambda_{\nu\kappa}} + {\color{blue}G_\nu}}
      \end{displaymath}
    \end{small}
  \item Magnetic Fields
    \begin{small}
      \begin{displaymath}
        \partial_t \paren{{\color{red}\detg} B^i}
        - \partial_j \sqrbrace{{\color{red}\detg}\paren{b^ju^i - b^i u^j}}
        = 0
      \end{displaymath}
    \end{small}
  \item Composition
    \begin{small}
      \begin{displaymath}
        \partial_t\paren{{\color{red}\detg}\rho_0 Y_e u^t}
        + \partial_i\paren{{\color{red}\detg}\rho_0Y_eu^i}
        = {\color{red}\detg} {\color{blue}G_{\text{ye}}}
      \end{displaymath}
    \end{small}
  \item Neutrino Transport
    \begin{small}
      \begin{displaymath}
        {\color{red}\frac{D}{d\lambda}}\paren{\frac{h^3\Inuf}{\eepsilon^3}}
        = \paren{\frac{h^2{\color{blue}\etanuf}}{\eepsilon^2}}
        - \paren{\frac{\eepsilon {\color{blue}\chinuf}}{h}} \paren{\frac{h^3\Inuf}{\eepsilon^3}},
      \end{displaymath}
    \end{small}
  \end{itemize}
\end{frame}

\begin{frame}
  \frametitle{Lessons from Flatland}
  \textbf{The Wind, With and Without Transport}
  \begin{columns}
    \begin{column}{6cm}
      \begin{center}
        %\includegraphics[width=0.9\textwidth]{ent4_rho/frame_0001}
        \animategraphics[width=0.9\columnwidth,every=10,autoplay,loop,controls]
        {5}{ent4_rho/frame_}{0001}{0388}
      \end{center}
    \end{column}
    \begin{column}{6cm}
      \begin{center}
        %\includegraphics[width=0.9\textwidth]{ent4_ye/frame_0001}
        \animategraphics[width=0.9\columnwidth,every=10,autoplay,loop,controls]
        {5}{ent4_ye/frame_}{0001}{0390}
      \end{center}
    \end{column}
  \end{columns}
\end{frame}

\begin{frame}
  \frametitle{Lessons from Flatland}
  \textbf{How Important are Interactions?}
  \begin{columns}
    \begin{column}{5cm}
      \begin{center}
        \includegraphics[width=0.9\textwidth]{dtau_tot_2d_1000M.pdf}
      \end{center}
    \end{column}
    \begin{column}{7cm}
        \includegraphics[width=0.7\textwidth]{dtau_integrated_tavg}\\
        \includegraphics[width=0.975\textwidth]{dtau_scatt_sadw_tavg}
    \end{column}
  \end{columns}
  \textbf{JMM}+, In prep.
\end{frame}

\begin{frame}
  \frametitle{Lessons From Flatland}
  \textbf{How does $Y_e$ change?}
  \begin{columns}
    \begin{column}{5cm}
      \begin{center}
        \includegraphics[width=\columnwidth]{Ye_divergence_plot_tavg}
      \end{center}
    \end{column}
    \begin{column}{7cm}
      \begin{center}
        \includegraphics[width=\columnwidth]{ye_zoH_tavg}
      \end{center}
    \end{column}
  \end{columns}
  \textbf{JMM}+, In prep.
\end{frame}

\begin{frame}
  \frametitle{GW170817, in Full 3D}
  \begin{columns}
    \begin{column}{6cm}
      \begin{center}
        \includegraphics[width=0.8\textwidth]{gw170817_dump_00000708-rho-xz.pdf}\\
        \includegraphics[width=0.8\textwidth]{gw170817_dump_00000708-rho-xy.pdf}
      \end{center}      
    \end{column}
    \begin{column}{6cm}
      \begin{center}
        \includegraphics[width=0.8\textwidth]{gw170817_dump_00000708-Ye-xz.pdf}\\
        \includegraphics[width=0.8\textwidth]{gw170817_dump_00000708-Ye-xy.pdf}
      \end{center}
    \end{column}
  \end{columns}
  Figures taken after $\sim$45ms. \textbf{JMM}+, in prep.
\end{frame}

\begin{frame}
  \frametitle{Outflow Properties}
  \begin{columns}
    \begin{column}{6.5cm}
      \begin{center}
        \includegraphics[width=0.9\columnwidth]{gw170817-ye-vs-theta}\\
        \includegraphics[width=0.9\columnwidth]{gw170817-ye-vs-time}
      \end{center}
    \end{column}
    \begin{column}{5.5cm}
      \begin{center}
        \includegraphics[width=0.9\columnwidth]{gw170817-ye-hist-mid-v-off-linear}\\
        \includegraphics[width=0.9\columnwidth]{gw170817-ye-2-times-hist-linear}
      \end{center}
    \end{column}
  \end{columns}
  \textbf{JMM}+, in prep.
\end{frame}

\begin{frame}
  \frametitle{Coming Soon:}
  \begin{itemize}
  \item Nucleosynthetic yields!
  \item Light curves!
  \item Direct comparison to observation!
  \end{itemize}
\end{frame}

\begin{frame}
  \frametitle{Summary}
  \begin{itemize}
  \item Neutron star mergers are the source of heavy elements in our
    universe!
    \begin{itemize}
    \item They're very cool and very difficult to model
    \item Many open questions. Where did the blue kilonova come from?
      Will we always see one?
    \item Need GRRMHD!
    \item Neutrino transport is critical!
    \end{itemize}
  \item We have developed $\nu\texttt{bhlight}${} to model these
    systems
  \item Preliminary results show disks can produce blue component of
    kilonova
  \item Lots more soon!
  \end{itemize}
\end{frame}

\backupbegin
\backupend

\end{document}