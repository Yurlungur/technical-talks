\documentclass[letter,nofootinbib,superscriptaddress,twocolumn]{revtex4-1}

\begin{document}

\title{Astrophysical Transients: What High Fidelity Supercomputer
  Simulations Can Teach Us About Matter at the Highest Densities and
  the Origin of Heavy Elements}

\author{Jonah M. Miller}
\email{jonahm@lanl.gov}
\affiliation{Center for Theoretical Astrophysics, Los Alamos National Laboratory, Los Alamos, NM, USA}
\affiliation{CCS-2, Los Alamos National Laboratory, Los Alamos, NM, USA}

\begin{abstract}

  The 2017 detection of the in-spiral and merger of two neutron stars
  was a landmark discovery in astrophysics. Through a wealth of
  multi-messenger data, we now know that the merger of these
  ultracompact stellar remnants is a central engine of short gamma ray
  bursts and a site of r-process nucleosynthesis, where the heaviest
  elements in our universe are formed. The radioactive decay of
  unstable heavy elements produced in such mergers powers an optical
  and infra-red transient: The kilonova.

  Along with core-collapse supernovae, neutron star mergers offer
  insight into the behavior of matter at incredibly high densities and
  temperatures. While pairwise interactions in this regime are well
  understood, collective behavior is not. This is the so-called
  high-temperature nuclear equation of state, and it is a grand
  challenge problem in nuclear physics.

  In this talk, I present my research program of nuclear astrophysics,
  where I use high-fidelity supercomputer simulations to investigate
  both the origin of heavy elements and the nuclear equation of state
  in core collapse supernovae, neutron star mergers, and their
  aftermath. I will discuss exciting results in this area synthesizing
  machine learning with observations, as well as recent progress on
  new modeling capabilities Phoebus and Parthenon.

\end{abstract}

\maketitle

\end{document}