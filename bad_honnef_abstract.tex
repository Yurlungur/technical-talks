\documentclass[letter,nofootinbib,superscriptaddress,twocolumn]{revtex4-1}

\begin{document}

\title{Impact of Neutrinos in Post-Merger Accretion Flows}

\author{Jonah M. Miller}
\email{jonahm@lanl.gov}
\affiliation{Center for Theoretical Astrophysics, Los Alamos National Laboratory, Los Alamos, NM, USA}
\affiliation{CCS-2, Los Alamos National Laboratory, Los Alamos, NM, USA}

\begin{abstract}

  The 2017 detection of the in-spiral and merger of two neutron stars
  was a landmark discovery in astrophysics. Through a wealth of
  multi-messenger data, we now know that the merger of these
  ultracompact stellar remnants is a central engine of short gamma ray
  bursts and a site of r-process nucleosynthesis, where the heaviest
  elements in our universe are formed. The radioactive decay of
  unstable heavy elements produced in such mergers powers an optical
  and infra-red transient: The kilonova.

  One key driver of nucleosynthesis and resultant electromagnetic
  afterglow is wind driven by an accretion disk formed around the
  compact remnant. Neutrino transport plays a key role in setting the
  electron fraction in this outflow, thus controlling the
  nucleosynthesis. I present recent progress in modeling this system,
  with an emphasis on uncertainties, degeneracies with the rest of the
  modeling process, and opportunities for the future.

\end{abstract}

\maketitle

\end{document}