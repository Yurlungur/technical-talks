\pdfminorversion=4
\documentclass[]{beamer}
\mode<presentation>
% Time-stamp: <2020-10-27 10:47:13 (jonahm)>

% beamer stuff
% Gives us the bottom line with all the goodies
\useoutertheme{infolines}
% Just the theme to use. Should be built into bemaer. Setting the
% height gets rid of a whole lot of whitespace
\usetheme[height=7mm]{Rochester}
\usefonttheme{serif}
% Usually beamer gives you navigation hyperlinks on the bottom
% right. I turned this off. It's annoying.
\setbeamertemplate{navigation symbols}{} 
% Makes my text boxes look pretty
\setbeamertemplate{blocks}[rounded][shadow=true] 
% Makes my bullet points 3d balls
\setbeamertemplate{items}[ball]

% Josh's packages
\usepackage{multimedia}
\usepackage{graphicx}

% Packages for me
\usepackage{amsmath,amssymb,latexsym,amsthm}
\usepackage[mathscr]{eucal}
\usepackage{mathrsfs}
\usepackage{verbatim}
\usepackage{braket}
\usepackage{listings}
\usepackage{xcolor}
% \usepackage[usenames,dvipsnames,svgnames,table]{xcolor}
\usepackage{fancybox}
\usepackage{animate}
% \usepackage{media9}
\usepackage{multicol}
\usepackage{mdframed}
\usepackage{hyperref}
%\usepackage{scalerel}
\usepackage[outline]{contour}
\contourlength{1.2pt}

% Macros

%Blackboard Bold
\newcommand{\R}{\mathbb{R}}
\newcommand{\Z}{\mathbb{Z}}
\newcommand{\N}{\mathbb{N}}
\newcommand{\Q}{\mathbb{Q}}
\newcommand{\A}{\mathbb{A}}
\newcommand{\E}{\mathbb{E}}
% other
\newcommand{\eval}{\biggr\rvert} %evaluated at
\newcommand{\myvec}[1]{\mathbf{#1}} % vectors for me
% total derivatives 
\newcommand{\diff}[2]{\frac{d #1}{d #2}} 
\newcommand{\dd}[1]{\frac{d}{d #1}}
% partial derivatives
\newcommand{\pd}[2]{\frac{\partial #1}{\partial #2}} 
\newcommand{\pdd}[1]{\frac{\partial}{\partial #1}} 
% Order operator
\DeclareRobustCommand{\orderof}{\ensuremath{\mathcal{O}}}

% braces
\newcommand{\paren}[1]{\left( #1 \right)}
\newcommand{\sqrbrace}[1]{\left[ #1 \right]}
\newcommand{\curlybrace}[1]{\left\{ #1 \right\}}
\newcommand{\inner}[1]{\paren{#1}}
\newcommand{\norm}[1]{\left| #1 \right|_2}

% g
\newcommand{\detg}{\sqrt{-g}}

% neutrinos
\newcommand{\eepsilon}{\epsilon} % energy
\newcommand{\fin}{f\in \{\nu_e,\nu_{\bar{e}},\nu_x\}}
\newcommand{\sign}{\text{sign}(f)}
\newcommand{\jnuf}{j_{\eepsilon,f}}
\newcommand{\etanuf}{\eta_{\eepsilon,f}}
\newcommand{\Inuf}{I_{\eepsilon,f}}
\newcommand{\chinuf}{\chi_{\eepsilon,f}}
\newcommand{\sigmanuf}{\sigma_{\eepsilon,f}}
\newcommand{\alphanuf}{\alpha_{\eepsilon,f}}
\newcommand{\numin}{\nu_{\text{min}}}
\newcommand{\numax}{\nu_{\text{max}}}


% tikz
\usepackage{tikz}
\usepackage{pgfplots}
\usetikzlibrary{arrows}
\usetikzlibrary{decorations.pathmorphing}

% Keys to support piece-wise uncovering of elements in TikZ pictures:
% \node[visible on=<2->](foo){Foo}
% \node[visible on=<{2,4}>](bar){Bar}   % put braces around comma expressions
% 
% Internally works by setting opacity=0 when invisible, which has the 
% adavantage (compared to \node<2->(foo){Foo} that the node is always there, hence
% always consumes space plus that coordinate (foo) is always available.
% 
% The actual command that implements the invisibility can be overriden
% by altering the style invisible. For instance \tikzsset{invisible/.style={opacity=0.2}}
% would dim the "invisible" parts. Alternatively, the color might be set to white, if the
% output driver does not support transparencies (e.g., PS) 
% 
\tikzset{
  invisible/.style={opacity=0},
  visible on/.style={alt={#1{}{invisible}}},
  alt/.code args={<#1>#2#3}{%
    \alt<#1>{\pgfkeysalso{#2}}{\pgfkeysalso{#3}} % \pgfkeysalso doesn't change the path
  },
}

% some nice flowchart features
\tikzset{
    mynode/.style={rectangle,rounded corners,draw=black, top color=white, bottom color=yellow!50,very thick, inner sep=1em, minimum size=3em, text centered},
    myarrow/.style={->, >=latex', shorten >=1pt, thick},
    mylabel/.style={text width=7em, text centered} 
}  

% squigly arrow
\tikzset{zigzag it/.style={decorate, decoration=zigzag}}

% define a really nice visible "purple"
\definecolor{gimppurple}{HTML}{AD26FB}
% a light grey
\definecolor{lightgrey}{HTML}{E0E0E0}
% for highlighting
\definecolor{deepblue}{rgb}{0,0,0.5}
\definecolor{deepred}{rgb}{0.6,0,0}
\definecolor{deepgreen}{rgb}{0,0.5,0}

% fonts
% Default fixed font does not support bold face
\DeclareFixedFont{\ttb}{T1}{txtt}{bx}{n}{12} % for bold
\DeclareFixedFont{\ttm}{T1}{txtt}{m}{n}{12}  % for normal

% Python style for highlighting
\newcommand\pythonstyle{\lstset{
language=Python,
basicstyle=\ttm,
otherkeywords={self},
keywordstyle=\ttb\color{deepblue},
emph={__init__},           
emphstyle=\ttb\color{deepred},
commentstyle=\ttfamily\color{deepred},
stringstyle=\color{deepgreen},
frame=tb,                     
showstringspaces=false        
}}

% Python environment
\lstnewenvironment{python}[1][]
{
\pythonstyle
\lstset{#1}
}
{}

\newcommand{\backupbegin}{
   \newcounter{finalframe}
   \setcounter{finalframe}{\value{framenumber}}
}
\newcommand{\backupend}{
   \setcounter{framenumber}{\value{finalframe}}
}

% Automatically generates section breaker slides
\AtBeginSection[]{
  \begin{frame}[plain]
  \vfill
  \centering
  \begin{beamercolorbox}[sep=8pt,center,shadow=true,rounded=true]{title}
    \usebeamerfont{title}\insertsectionhead\par%
  \end{beamercolorbox}
  \vfill
  \end{frame}
}

\title[Post-merger Disks]{Challenges and Opportunities in Post-Merger Disks}
% \subtitle{Models and Implications}
\author[J. Miller]{\textcolor{blue}{Jonah M. Miller},
  C. Fryer, M. Mumpower, R. Surman\\
And Many More...}
\institute[LANL]{Los Alamos National Laboratory}
% \titlegraphic{\vspace{1cm}}
% \titlegraphic{\includegraphics[height=0.25\textheight]{3d_render}}
% \date[9/19/19]{LA Astro Seminar}
\date[ICERM]{ICERM Workshop}

\graphicspath{{figures/}}

\begin{document}

% \begin{frame}[plain]
%     \tikz [remember picture, overlay] 
%     \node at ([xshift=2cm,yshift=-2cm]current page.west)
%     {\includegraphics[width=0.25\textwidth,clip,trim={150 0 150 0}]{3d_render}};
%     \tikz [remember picture, overlay] 
%     \node at ([xshift=-2cm,yshift=-2cm]current page.east)
%     {\includegraphics[width=0.25\textwidth,clip,trim={0 0 0 0}]{visit0006-gimp3}};
%   \titlepage
% \end{frame}

\begin{frame}
  % \frametitle{Neutron Star Mergers: A 2+ Component Model}
  \frametitle{Challenges and Opportunities in Post-Merger Disks}
  \resizebox{12cm}{!}{
    \begin{tikzpicture}
      \coordinate (origin) at (0,0);
      \pgfmathsetmacro{\dbx}{0.5}
      \pgfmathsetmacro{\dby}{0.05}
      \pgfmathsetmacro{\dex}{2.}
      \pgfmathsetmacro{\dey}{0.25}
      \pgfmathsetmacro{\dcc}{2.1}
      \pgfmathsetmacro{\tcx}{5.0}
      
      \foreach \i in {-1,1}
      {
        \fill[ball color=blue] (0, \i*2.9) ellipse (2 and 3);
      }
      
      \foreach \i in {-1,1}
      {
        % disk
        \fill[color=orange]
        (\i*\dbx,\dby) -- (\i*\dex,\dey)
        .. controls (\i*\dcc,0) .. (\i*\dex,-\dey)
        -- (\i*\dbx,-\dby) -- cycle;
        
        % tidal ejecta
        \fill[color=red] (\i*\tcx,0) ellipse (0.75 and 1.5);
      }
      
      % viewing
      \draw[dashed,ultra thick,black] (origin) -- (5,4);
      \draw[dashed,ultra thick,black] (origin) -- (3,6);
      \fill[color=gray,opacity=0.5] (origin) -- (5,4) -- (3,6) -- cycle;
      \node[right,align=left] at (3.5,5)
      {\Large $\sim$ GW170817\\ \Large Viewing angle};
      
      % bh
      \shade[ball color=black] (origin) circle (0.25);
      
      % text
      \draw[<-,red, ultra thick] (\tcx,1.5)
      -- ++(0.,0.5) -- ++(0.5,0)
      node[right,align=left]
      {\Large \color{red}$n$-rich\\\Large Tidal Ejecta\\ \Large $\sim 5\times 10^{-3}M_{\odot}$};
      
      \draw[<-,blue, ultra thick] (0,6) -- ++(0,1) -- ++(-2,0)
      node[left,align=right]
      {\Large \color{blue}$n$-poor\\\Large wind\\\Large $\sim 10^{-2}M_{\odot}$};
      
      \draw[<-,orange, ultra thick] (\dex,-\dey)
      -- ++(2,-3) -- ++(1,0)
      node[right,align=left]
      {\Large \color{orange}Disk\\\Large $\sim 10^{-1}M_{\odot}$};
      
      \draw[<-,black, ultra thick] (-0.25,-0.25)
      -- ++(-2,-3) -- ++(-1,0)
      node[left,align=right]
      {\Large \color{black}Remnant\\ \Large $\sim 2 M_{\odot}$};
      \node[visible on=<1>,inner sep=0pt] at (15,4) {
        \includegraphics[width=0.7\columnwidth,clip,trim={150 0 150 0}]{3d_render}
      };
      \node[visible on=<1>,inner sep=0pt] at (14,-4) {
        \includegraphics[width=\columnwidth]{gw170817_dump_00000708-Ye-xz}
      };
      \node[visible on=<2>,inner sep=0pt] at (15,0) {
        \includegraphics[width=\columnwidth]{gw170817-sinh10-divergence-rates-tavg}
      };
      % \node[visible on=<2>,inner sep=0pt] at (0,0) {
      %   \includegraphics[width=\columnwidth]{gw170817-ye-vs-theta-folded-5}
      % };
    \end{tikzpicture}
  }
\end{frame}

\begin{frame}
  \frametitle{Challenges and Opportunities in Post-Merger Disks}
  \begin{columns}
    \begin{column}{6cm}
      \begin{itemize}
      \item Need all of:
        \begin{itemize}
        \item GR
        \item MHD
        \item Neutrino transport!
        \end{itemize}
      \item Approximate treatments of neutrinos not sufficient
      \item Magnetic field, which powers MHD turbulence is poorly understood
      \item Disentangling different parts of the engine in observable
        signatures very challenging
      \item Long-term evolution, especially with a central NS, rather
        than BH, extremely difficult
      \end{itemize}
    \end{column}
    \begin{column}{6cm}
      \begin{center}
        \includegraphics[width=\columnwidth]{gw170817-sinh10-divergence-rates-tavg}
      \end{center}
    \end{column}
  \end{columns}
\end{frame}

\begin{frame}
  \frametitle{MOCMC}
  \begin{columns}
    \begin{column}{6cM}
      \begin{center}
        {\huge\color{blue}\underline{Moments}}
        $$\paren{R^t_{\ \nu}}_{;t} = -\paren{R^i_{\ \nu}}_{;i} - G_\nu$$
      \end{center}
    \end{column}
    \begin{column}{6cM}
      \begin{center}
        {\huge\color{red}\underline{+Samples}}
        $$\frac{D}{d\lambda} \paren{\frac{I_\nu}{\nu^3}} = \frac{j_\nu}{\nu^2} - (\nu \alpha_\nu)\paren{\frac{I_\nu}{\nu^3}} $$
        \end{center}
    \end{column}
  \end{columns}
  \vspace{0.5cm}
  \begin{columns}
    \begin{column}{6cm}
      \begin{itemize}
      \item Moment system closed by \textit{deterministic} long-characteristic system
      \item $\sim O(N)$ convergence for full frequency-dependent transport
      \item Efficient and stable at large optical depths and radiation pressures
      \item Straightforward local adaptivity
      \end{itemize}
    \end{column}
    \begin{column}{6cm}
      \resizebox{\columnwidth}{!}{
        \begin{tikzpicture}
          \node[visible on=<1>,inner sep=0pt] at (0,0)
          {\includegraphics[width=6cm]{mocmc-hohlraum}};
          \node[visible on=<2>,inner sep=0pt] at (0,0)
          {\includegraphics[width=6cm]{farrisshock}};
        \end{tikzpicture}
      }
    \end{column}
  \end{columns}
\end{frame}

\backupbegin
\backupend

\end{document}