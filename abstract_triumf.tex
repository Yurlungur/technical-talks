\documentclass[letter,nofootinbib,superscriptaddress,twocolumn]{revtex4-1}

\begin{document}

\title{Neutrinos, Collapsars, Neutron Star Mergers, and You: How neutrinos impact heavy element formation in the universe, and the resulting electromagnetic counterpart}

\author{Jonah M. Miller}
\email{jonahm@lanl.gov}
\affiliation{Center for Theoretical Astrophysics, Los Alamos National Laboratory, Los Alamos, NM, USA}
\affiliation{CCS-2, Los Alamos National Laboratory, Los Alamos, NM, USA}

\begin{abstract}

  The 2017 detection of the in-spiral and merger of two neutron stars
  was a landmark discovery in astrophysics. Through a wealth of
  multi-messenger data, we now know that the merger of these
  ultracompact stellar remnants is a central engine of short gamma ray
  bursts and a site of r-process nucleosynthesis, where the heaviest
  elements in our universe are formed. The radioactive decay of
  unstable heavy elements produced in such mergers powers an optical
  and infra-red transient: The kilonova.

  One key driver of nucleosynthesis and resultant electromagnetic
  afterglow is wind driven by an accretion disk formed around the
  compact remnant. Neutrino transport plays a key role in setting the
  electron fraction in this outflow, thus controlling the
  nucleosynthesis.

  Collapsars are black hole accretion disks formed after the core of a
  massive, rapidly rotating star collapses to a black hole. These
  dramatic systems rely on much the same physics and modeling as
  post-merger disks, and can also be a key driver of r-processes
  nucleosynthesis.

  I present recent progress in modeling these enigmatic systems, with
  an emphasis on both the impact and techniques of detailed neutrino
  transport.

\end{abstract}

\maketitle

\end{document}