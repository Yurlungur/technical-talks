%\pdfminorversion=4
\documentclass[]{beamer}
\mode<presentation>
% Time-stamp: <2017-10-17 12:46:02 (jonahm)>

% beamer stuff
% Gives us the bottom line with all the goodies
\useoutertheme{infolines}
% Just the theme to use. Should be built into bemaer. Setting the
% height gets rid of a whole lot of whitespace
\usetheme[height=7mm]{Rochester}
\usefonttheme{serif}
% Usually beamer gives you navigation hyperlinks on the bottom
% right. I turned this off. It's annoying.
\setbeamertemplate{navigation symbols}{} 
% Makes my text boxes look pretty
\setbeamertemplate{blocks}[rounded][shadow=true] 
% Makes my bullet points 3d balls
\setbeamertemplate{items}[ball]

% Josh's packages
\usepackage{multimedia}
\usepackage{graphicx}

% Packages for me
\usepackage{amsmath,amssymb,latexsym,amsthm}
\usepackage[mathscr]{eucal}
\usepackage{mathrsfs}
\usepackage{verbatim}
\usepackage{braket}
\usepackage{listings}
\usepackage{xcolor}
% \usepackage[usenames,dvipsnames,svgnames,table]{xcolor}
\usepackage{fancybox}
\usepackage{animate}
% \usepackage{media9}
\usepackage{multicol}
\usepackage{mdframed}
\usepackage{hyperref}
%\usepackage{scalerel}
\usepackage[outline]{contour}
\contourlength{1.2pt}

% Macros

%Blackboard Bold
\newcommand{\R}{\mathbb{R}}
\newcommand{\Z}{\mathbb{Z}}
\newcommand{\N}{\mathbb{N}}
\newcommand{\Q}{\mathbb{Q}}
\newcommand{\A}{\mathbb{A}}
\newcommand{\E}{\mathbb{E}}
% other
\newcommand{\eval}{\biggr\rvert} %evaluated at
\newcommand{\myvec}[1]{\mathbf{#1}} % vectors for me
% total derivatives 
\newcommand{\diff}[2]{\frac{d #1}{d #2}} 
\newcommand{\dd}[1]{\frac{d}{d #1}}
% partial derivatives
\newcommand{\pd}[2]{\frac{\partial #1}{\partial #2}} 
\newcommand{\pdd}[1]{\frac{\partial}{\partial #1}} 
% Order operator
\DeclareRobustCommand{\orderof}{\ensuremath{\mathcal{O}}}

% tikz
\usepackage{tikz}
\usetikzlibrary{arrows}
\usetikzlibrary{decorations.pathmorphing}
\usepackage{pgfplots}


% Keys to support piece-wise uncovering of elements in TikZ pictures:
% \node[visible on=<2->](foo){Foo}
% \node[visible on=<{2,4}>](bar){Bar}   % put braces around comma expressions
% 
% Internally works by setting opacity=0 when invisible, which has the 
% adavantage (compared to \node<2->(foo){Foo} that the node is always there, hence
% always consumes space plus that coordinate (foo) is always available.
% 
% The actual command that implements the invisibility can be overriden
% by altering the style invisible. For instance \tikzsset{invisible/.style={opacity=0.2}}
% would dim the "invisible" parts. Alternatively, the color might be set to white, if the
% output driver does not support transparencies (e.g., PS) 
% 
\tikzset{
  invisible/.style={opacity=0},
  visible on/.style={alt={#1{}{invisible}}},
  alt/.code args={<#1>#2#3}{%
    \alt<#1>{\pgfkeysalso{#2}}{\pgfkeysalso{#3}} % \pgfkeysalso doesn't change the path
  },
}

% some nice flowchart features
\tikzset{
    mynode/.style={rectangle,rounded corners,draw=black, top color=white, bottom color=yellow!50,very thick, inner sep=1em, minimum size=3em, text centered},
    myarrow/.style={->, >=latex', shorten >=1pt, thick},
    mylabel/.style={text width=7em, text centered} 
}  

% squigly arrow
\tikzset{zigzag it/.style={decorate, decoration=zigzag}}

\newcommand{\PI}{3.14}

% define a really nice visible "purple"
\definecolor{gimppurple}{HTML}{AD26FB}
% a light grey
\definecolor{lightgrey}{HTML}{E0E0E0}
% for highlighting
\definecolor{deepblue}{rgb}{0,0,0.5}
\definecolor{deepred}{rgb}{0.6,0,0}
\definecolor{deepgreen}{rgb}{0,0.5,0}

% fonts
% Default fixed font does not support bold face
\DeclareFixedFont{\ttb}{T1}{txtt}{bx}{n}{12} % for bold
\DeclareFixedFont{\ttm}{T1}{txtt}{m}{n}{12}  % for normal

% Python style for highlighting
\newcommand\pythonstyle{\lstset{
language=Python,
basicstyle=\ttm,
otherkeywords={self},
keywordstyle=\ttb\color{deepblue},
emph={__init__},           
emphstyle=\ttb\color{deepred},
commentstyle=\ttfamily\color{deepred},
stringstyle=\color{deepgreen},
frame=tb,                     
showstringspaces=false        
}}

% Python environment
\lstnewenvironment{python}[1][]
{
\pythonstyle
\lstset{#1}
}
{}

\newcommand{\backupbegin}{
   \newcounter{finalframe}
   \setcounter{finalframe}{\value{framenumber}}
}
\newcommand{\backupend}{
   \setcounter{framenumber}{\value{finalframe}}
}

% Automatically generates section breaker slides
\AtBeginSection[]{
  \begin{frame}[plain]
  \vfill
  \centering
  \begin{beamercolorbox}[sep=8pt,center,shadow=true,rounded=true]{title}
    \usebeamerfont{title}\insertsectionhead\par%
  \end{beamercolorbox}
  \vfill
  \end{frame}
}

\title[$\nu$bhlight]{Neutron Star Mergers and Neutrino Driven Accretion Flows} 
\author[J. Miller]{\textbf{Jonah M. Miller}\\B. Ryan, J. Dolence, C. Fryer, A. Burrows}
\institute[LANL]{Los Alamos National Lab}

\date[18/07/18]{Astro Coffee at the IAS\\LA-UR-18-26592}

\graphicspath{{figures/}}

\begin{document}

\begin{frame}[plain]
\titlepage
\end{frame}

\begin{frame}
  \frametitle{GW170817}
  \begin{center}
    \href{https://youtu.be/e7LcmWiclOs}{\includegraphics[height=6cm]{GW170817-rendition}}
  \end{center}
  \url{https://youtu.be/e7LcmWiclOs}\\
  NSF/LIGO/Sonoma State University/A. Simonnet\\
   NASA's Goddard Space Flight Center/CI Lab
\end{frame}

\begin{frame}
  \frametitle{Observations Galore!}
  \begin{center}
    \includegraphics[height=7cm]{abbot-timeline}
  \end{center}
  Abbot+, 2017
\end{frame}

\begin{frame}
  \frametitle{Merger Dynamics in Short}
  \begin{center}
    %\includegraphics[height=0.8\textheight]{frames/betabin_000}
    \animategraphics[height=0.8\textheight,every=10,autoplay,loop,controls]
    {5}{frames/betabin_}{000}{374}
  \end{center}
    \textbf{JMM}+, 2016
\end{frame}

\begin{frame}
  \frametitle{Disk Drives a Jet}
  \begin{center}
    \includegraphics[width=0.9\textwidth]{ruiz-2016}
  \end{center}
  Ruiz+, 2016
\end{frame}

\begin{frame}
  \frametitle{The Radioactive Cloud}
  \begin{center}
    \includegraphics[height=7cm,clip,trim={4cm 1.5cm 4cm 0cm}]
      {ejecta-morphology-z-projection}
  \end{center}
  \textbf{JMM}+, 2016
\end{frame}

\begin{frame}
  \frametitle{The Kilonova}
  \begin{center}
    \includegraphics[width=10cm]{swope-image}
  \end{center}
  M2H/UC Santa Cruz and Carnegie Observatories/Ryan Foley
\end{frame}

\begin{frame}
  \frametitle{Infrared Light Curves}
  \begin{center}
    \includegraphics[height=7cm]{tanvir_light_curve}
  \end{center}
  Tanvir et al., 2017/HST
\end{frame}

\begin{frame}
  \frametitle{Infrared Light Curves}
  \begin{center}
    \includegraphics[width=0.99\textwidth]{tanvir_light_curve_model}
  \end{center}
  Tanvir et al., 2017/HST
\end{frame}

\begin{frame}
  \frametitle{Blue Kilonova Comes from the Disk?}
  \begin{center}
    \includegraphics[width=7cm]{disk_3d_render}
  \end{center}
  \textbf{JMM}+, in prep.
\end{frame}

\begin{frame}
  \frametitle{Wind Dynamics? Nucleosynthetic Yield? Jet Physics?}
  \begin{center}
    %\includegraphics[height=0.8\textheight]{torus2d_frames/frame_0400}
    \animategraphics[height=0.8\textheight,every=10,autoplay,loop,controls]
    {5}{torus2d_frames/frame_}{0000}{0400}
  \end{center}
    \textbf{JMM}+, in prep.
\end{frame}

\begin{frame}
  \frametitle{Presenting $\nu$bhlight!}
  \begin{itemize}
  \item $\nu$bhlight is a radiation GRMHD code.
  \end{itemize}
  \begin{columns}
    \begin{column}{6cm}
      \begin{itemize}
      \item Physics:
        \begin{itemize}
        \item General relativity (static spacetimes)
        \item Magnetic fields
        \item Fluid dynamics
        \item Realistic nuclear equations of state
        \item Neutrino transport and weak interactions
        \end{itemize}
      \end{itemize}
    \end{column}
    \begin{column}{6cm}
      \begin{itemize}
      \item Implementation:
        \begin{itemize}
        \item Finite volume scheme for the fluid
        \item Constrained transport for the $B$-field
        \item Monte Carlo for $\nu$
        \item Tabulated microphysics
        \item Built on Gammie+ (2003), Dolence+ (2009),  and Ryan+  (2015)
        \end{itemize}
      \end{itemize}
    \end{column}
  \end{columns}
\end{frame}

\begin{frame}
  \frametitle{Code Tests}
  \begin{columns}
    \begin{column}{6cm}
      \centering
      \includegraphics[height=7cm]{fornax-equil-comparison}
    \end{column}
    \begin{column}{6cm}
      \centering
      \includegraphics[height=3.5cm]{multiscatt}\\
      \includegraphics[height=3.5cm]{yedecay_electron_nu_ye}
    \end{column}
  \end{columns}
  \textbf{JMM}+, in prep.
\end{frame}

\begin{frame}
  \frametitle{(Very!) Preliminary Results}
  \begin{columns}
    \begin{column}{8cm}
      \begin{center}
       % \includegraphics[width=\columnwidth]{torus_nu_frames/frame_0400}
        \animategraphics[width=\columnwidth,every=10,autoplay,loop,controls]
        {5}{torus_nu_frames/frame_}{0000}{0400}
      \end{center}
    \end{column}
    \begin{column}{4cm}
      \begin{center}
        %\includegraphics[width=\columnwidth]{torus_ye_frames/frame_0400}
        \animategraphics[width=\columnwidth,every=10,autoplay,loop,controls]
        {5}{torus_ye_frames/frame_}{0000}{0400}
        \begin{itemize}
          \item \textbf{JMM}+, in prep. 
        \end{itemize}
      \end{center}
    \end{column}
  \end{columns}
\end{frame}

\begin{frame}
  \frametitle{Conclusion}
  \begin{itemize}
  \item The detection of a binary neutron star is very exciting!
  \item There are open questions about the central engine:
    \begin{itemize}
    \item How important is the disk wind for nucleosynthesis and the kilonova?
    \item What drives the wind? MHD? Neutrinos?
    \item What drives the jet physics?
    \end{itemize}
    \item Requires detailed radiation GRMHD and nucleosynthesis calculations
    \item We present $\nu$bhlight as a tool for modeling the central engine
  \end{itemize}
\end{frame}

\backupbegin
\backupend

\end{document}