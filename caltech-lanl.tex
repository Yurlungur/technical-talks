\pdfminorversion=4
\documentclass[]{beamer}
\mode<presentation>

% beamer stuff
% Gives us the bottom line with all the goodies
\useoutertheme{infolines}
% Just the theme to use. Should be built into bemaer. Setting the
% height gets rid of a whole lot of whitespace
\usetheme[height=7mm]{Rochester}
\usefonttheme{serif}
% Usually beamer gives you navigation hyperlinks on the bottom
% right. I turned this off. It's annoying.
\setbeamertemplate{navigation symbols}{} 
% Makes my text boxes look pretty
\setbeamertemplate{blocks}[rounded][shadow=true] 
% Makes my bullet points 3d balls
\setbeamertemplate{items}[ball]

% Josh's packages
\usepackage{multimedia}
\usepackage{graphicx}

% Packages for me
\usepackage{amsmath,amssymb,latexsym,amsthm}
\usepackage[mathscr]{eucal}
\usepackage{mathrsfs}
\usepackage{verbatim}
\usepackage{braket}
\usepackage{listings}
\usepackage{xcolor}
% \usepackage[usenames,dvipsnames,svgnames,table]{xcolor}
\usepackage{fancybox}
\usepackage{animate}
% \usepackage{media9}
\usepackage{multicol}
\usepackage{mdframed}
\usepackage{hyperref}
%\usepackage{scalerel}
\usepackage[outline]{contour}
\contourlength{1.2pt}

% Macros

%Blackboard Bold
\newcommand{\R}{\mathbb{R}}
\newcommand{\Z}{\mathbb{Z}}
\newcommand{\N}{\mathbb{N}}
\newcommand{\Q}{\mathbb{Q}}
\newcommand{\A}{\mathbb{A}}
\newcommand{\E}{\mathbb{E}}
% other
\newcommand{\eval}{\biggr\rvert} %evaluated at
\newcommand{\myvec}[1]{\mathbf{#1}} % vectors for me
% total derivatives 
\newcommand{\diff}[2]{\frac{d #1}{d #2}} 
\newcommand{\dd}[1]{\frac{d}{d #1}}
% partial derivatives
\newcommand{\pd}[2]{\frac{\partial #1}{\partial #2}} 
\newcommand{\pdd}[1]{\frac{\partial}{\partial #1}} 
% Order operator
\DeclareRobustCommand{\orderof}{\ensuremath{\mathcal{O}}}

% braces
\newcommand{\paren}[1]{\left( #1 \right)}
\newcommand{\sqrbrace}[1]{\left[ #1 \right]}
\newcommand{\curlybrace}[1]{\left\{ #1 \right\}}
\newcommand{\inner}[1]{\paren{#1}}
\newcommand{\norm}[1]{\left| #1 \right|_2}

% g
\newcommand{\detg}{\sqrt{-g}}

% neutrinos
\newcommand{\eepsilon}{\epsilon} % energy
\newcommand{\fin}{f\in \{\nu_e,\nu_{\bar{e}},\nu_x\}}
\newcommand{\sign}{\text{sign}(f)}
\newcommand{\jnuf}{j_{\eepsilon,f}}
\newcommand{\etanuf}{\eta_{\eepsilon,f}}
\newcommand{\Inuf}{I_{\eepsilon,f}}
\newcommand{\chinuf}{\chi_{\eepsilon,f}}
\newcommand{\sigmanuf}{\sigma_{\eepsilon,f}}
\newcommand{\alphanuf}{\alpha_{\eepsilon,f}}
\newcommand{\numin}{\nu_{\text{min}}}
\newcommand{\numax}{\nu_{\text{max}}}


% tikz
\usepackage{tikz}
\usepackage{pgfplots}
\usetikzlibrary{calc,fadings,decorations.pathreplacing}
\usetikzlibrary{arrows}
\usetikzlibrary{decorations.pathmorphing}
\usetikzlibrary{decorations.markings}
\usetikzlibrary{arrows.meta,bending}

% Keys to support piece-wise uncovering of elements in TikZ pictures:
% \node[visible on=<2->](foo){Foo}
% \node[visible on=<{2,4}>](bar){Bar}   % put braces around comma expressions
% 
% Internally works by setting opacity=0 when invisible, which has the 
% adavantage (compared to \node<2->(foo){Foo} that the node is always there, hence
% always consumes space plus that coordinate (foo) is always available.
% 
% The actual command that implements the invisibility can be overriden
% by altering the style invisible. For instance \tikzsset{invisible/.style={opacity=0.2}}
% would dim the "invisible" parts. Alternatively, the color might be set to white, if the
% output driver does not support transparencies (e.g., PS) 
% 
\tikzset{
  invisible/.style={opacity=0},
  visible on/.style={alt={#1{}{invisible}}},
  alt/.code args={<#1>#2#3}{%
    \alt<#1>{\pgfkeysalso{#2}}{\pgfkeysalso{#3}} % \pgfkeysalso doesn't change the path
  },
}

% some nice flowchart features
\tikzset{
    mynode/.style={rectangle,rounded corners,draw=black, top color=white, bottom color=yellow!50,very thick, inner sep=1em, minimum size=3em, text centered},
    myimgnode/.style={rectangle,rounded corners,draw=black, top color=white, bottom color=yellow!50,very thick, inner sep=0.25em, minimum size=3em, text centered},
    myarrow/.style={->, >=latex', shorten >=2pt, thick},
    mylabel/.style={text width=8em, text centered} 
}  

% squigly arrow
\tikzset{zigzag it/.style={decorate, decoration=zigzag}}

% Color morphing arrow
\tikzset{colormorph/.style n args={3}{
    postaction={
    decorate,
    decoration={
    markings,
    mark=between positions 0 and \pgfdecoratedpathlength step 0.5pt with {
    \pgfmathsetmacro\myval{multiply(
        divide(
        \pgfkeysvalueof{/pgf/decoration/mark info/distance from start}, \pgfdecoratedpathlength
        ),
        100
    )};
    \pgfsetfillcolor{#3!\myval!#2};
    \pgfpathcircle{\pgfpointorigin}{#1};
    \pgfusepath{fill};}
}}}}

% define a really nice visible "purple"
\definecolor{gimppurple}{HTML}{AD26FB}
% a light grey
\definecolor{lightgrey}{HTML}{E0E0E0}
% for highlighting
\definecolor{deepblue}{rgb}{0,0,0.5}
\definecolor{deepred}{rgb}{0.6,0,0}
\definecolor{deepgreen}{rgb}{0,0.5,0}

% fonts
% Default fixed font does not support bold face
\DeclareFixedFont{\ttb}{T1}{txtt}{bx}{n}{12} % for bold
\DeclareFixedFont{\ttm}{T1}{txtt}{m}{n}{12}  % for normal

% Python style for highlighting
\newcommand\pythonstyle{\lstset{
language=Python,
basicstyle=\ttm,
otherkeywords={self},
keywordstyle=\ttb\color{deepblue},
emph={__init__},           
emphstyle=\ttb\color{deepred},
commentstyle=\ttfamily\color{deepred},
stringstyle=\color{deepgreen},
frame=tb,                     
showstringspaces=false        
}}

% Python environment
\lstnewenvironment{python}[1][]
{
\pythonstyle
\lstset{#1}
}
{}

\newcommand{\backupbegin}{
   \newcounter{finalframe}
   \setcounter{finalframe}{\value{framenumber}}
}
\newcommand{\backupend}{
   \setcounter{framenumber}{\value{finalframe}}
}

% Automatically generates section breaker slides
\AtBeginSection[]{
  \begin{frame}[plain]
  \vfill
  \centering
  \begin{beamercolorbox}[sep=8pt,center,shadow=true,rounded=true]{title}
    \usebeamerfont{title}\insertsectionhead\par%
  \end{beamercolorbox}
  \vfill
  \end{frame}
}

\title[Kilonovae]{Kilonova Modeling}
% \subtitle{Models and Implications}
\author[J. Miller]{\textcolor{blue}{Jonah M. Miller}, \color{black}{Ryan Wollaeger,}\\
  \color{black}{And Many More...}}
\institute[LANL]{Los Alamos National Laboratory}
% \titlegraphic{\vspace{1cm}}
% \titlegraphic{\includegraphics[height=0.25\textheight]{3d_render}}
% \date[9/19/19]{LA Astro Seminar}
\date[Caltech-LANL]{Caltech-LANL Mini Conference}

\graphicspath{{figures/}}

\begin{document}

\begin{frame}[plain]
    \tikz [remember picture, overlay] 
    \node at ([xshift=2cm,yshift=-2cm]current page.west)
    {\includegraphics[width=0.25\textwidth,clip,trim={150 0 150 0}]{3d_render}};
    \tikz [remember picture, overlay] 
    \node at ([xshift=-2cm,yshift=-2cm]current page.east)
    {\includegraphics[width=0.25\textwidth,clip,trim={0 0 0 0}]{visit0006-gimp3}};
  \titlepage
\end{frame}

\begin{frame}[plain]
  \begin{itemize}
  \item This Document cleared for unlimited release with LA-UR-XX-XXXX
  \item The submitted materials have been authored by an employee or
    employees of Triad National Security, LLC (Triad) under contract
    with the U.S.  Department of Energy/National Nuclear Security
    Administration (DOE/NNSA).  Accordingly, the U.S. Government
    retains an irrevocable, nonexclusive, royalty- free license to
    publish, translate, reproduce, use, or dispose of the published
    form of the work and to authorize others to do the same for
    U.S. Government purposes.”
  \end{itemize}
\end{frame}

\begin{frame}
  \frametitle{The Makings of a Kilonova}
  \begin{itemize}
  \item {\color{red}Duration/relevant time scales}
  \item {\color{blue}Methods}
  \end{itemize}
  \resizebox{12cm}{!}{
    \begin{tikzpicture}[node distance=9em]
      \node[myimgnode] (merger) {\includegraphics[width=6em]{frames/betabin_100}};
      \node[myimgnode, right of=merger] (disk) {\includegraphics[width=6em]{disk_image_no_text}};
      \node[myimgnode, right of=disk] (nuc) {\includegraphics[width=6em]{skynet_ye_0p13/frame_0108}};
      \node[myimgnode, right of=nuc] (rad) {\includegraphics[width=6em]{spectral_evolution}};
      \node[mylabel, below of=merger] (merger label)
      {\small In-spiral:\\
        \color{red}$\sim$ $\mu$s$\to$30 s\\
        \color{blue}numerical relativity\\
        \color{black}ET, SpEC, etc.
      };
      \node[mylabel, below of=disk] (disk label)
      {\small Disk+GRB:\\
        \color{red}$\sim$ $\mu$s$\to$1 s\\
        \color{blue}GR$\nu$RMHD\\
        \color{black}$\nu${\tt bhlight, HARM}
      };
      \node[mylabel, below of=nuc] (nuc label)
      {\small Non-equilibrium reactions:\\
        \color{red}$\sim$ s$\to$hours$\to$Myrs\\
        \color{blue}Reaction network\\
        \color{black}PRISM, SkyNet
      };
      \node[mylabel, below of=rad] (rad label)
      {\small Photon transport:\\
        \color{red}$\sim$ Weeks\\
        \color{blue}DDMC-IMC\\
        \color{black}{\tt SuperNu,sedona}
      };

      \draw[myarrow] (merger) -- (disk);
      \draw[myarrow] (disk) -- (nuc);
      \draw[myarrow] (nuc) -- (rad);
      \draw[myarrow] (merger label) -- (merger);
      \draw[myarrow] (disk label) -- (disk);
      \draw[myarrow] (nuc label) -- (nuc);
      \draw[myarrow] (rad label) -- (rad);
    \end{tikzpicture}
  }
\end{frame}

\begin{frame}
  \frametitle{Ingredients In Kilonova Disk Modeling}
  \begin{itemize}
  \item General relativity
    \begin{itemize}
    \item Rotating black hole spacetime
    \end{itemize}
  \item Plasma physics
    \begin{itemize}
    \item Ideal magnetohydrodynamics
    \end{itemize}
  \item Nuclear physics
    \begin{itemize}
    \item Hot gas treated as being in nuclear-statistical equilibrium via \textbf{equation of state}
    \item Cooling outflow treated in postprocessing via \textbf{nuclear reaction networks}
    \end{itemize}
  \item Radiation physics
    \begin{itemize}
    \item Material is opaque to photons, can be incorporated in plasma physics
    \item Material \textit{not} opaque to \textbf{neutrinos}.
    \item Neutrinos can \textit{change the composition of the
        material} by converting neutrons to protons and vice versa.
    \end{itemize}
  \end{itemize}
\end{frame}


\begin{frame}
  \frametitle{The LANL Disk Codes}
  \begin{itemize}
  \item $\nu\texttt{bhlight}$~General relativistic radiation magnetohydrodynamics for
    kilonova disks
    \begin{itemize}
    \item To-date, still the most advanced capability in the world for
      post-merger accretion disks. Highlight is Monte Carlo neutrino
      transport coupled to GRMHD constrained transport finite volumes.
    \item Open Source! {\color{blue}\url{https://github.com/LANL/nubhlight}}
    \end{itemize}
  \item Phoebus-Performance-portable relativistic astrophysics
    \begin{itemize}
    \item Includes $\nu\texttt{bhlight}$ physics on GPUs, with AMR.
    \item Also contains self-gravity, different transport treatments.
    \item Part of a software stack built up over the last several
      years, including equations of state, performance portable AMR,
      etc.
    \item {\color{blue}\url{https://github.com/parthenon-hpc-lab/parthenon}}
    \item {\color{blue}\url{https://github.com/lanl/phoebus}}
    \item {\color{blue}\url{https://github.com/lanl/singularity-eos}}
    \item {\color{blue}\url{https://github.com/lanl/singularity-opac}}
    \item {\color{blue}\url{https://github.com/lanl/spiner}}
    \end{itemize}
  \end{itemize}
\end{frame}

\begin{frame}
  \frametitle{The August 2017 Disk}
  \begin{center}
    % \includegraphics[height=0.475\textheight]{gw170817disk_Ye_close/frame_0831} \\
    % \includegraphics[height=0.475\textheight]{gw170817disk_Ye_far/frame_0831} 
    \animategraphics[height=0.475\textheight,every=5,autoplay,loop,controls=off]
    {10}{gw170817disk_Ye_close/frame_}{0001}{0837} \\
    \animategraphics[height=0.475\textheight,every=5,autoplay,loop,controls=off]
    {10}{gw170817disk_Ye_far/frame_}{0001}{0837} 
  \end{center}
  %\textbf{JMM}+, in prep.
\end{frame}

\begin{frame}
  \frametitle{Outflow}
  \begin{columns}
    \begin{column}{6cm}
      \includegraphics[height=0.9\textheight]{gw170817-ye-vs-theta-folded-5}\\
      \begin{tiny}
        \textbf{JMM} et al. PRD \textbf{100} 023008 (2019)
      \end{tiny}
    \end{column}
    \begin{column}{6cm}
      \begin{center}
        \includegraphics[width=\columnwidth]{gw170817-yields-2}
      \end{center}
      \begin{tiny}
        \textbf{JMM} et al. PRD \textbf{100} 023008 (2019)
      \end{tiny}
    \end{column}    
  \end{columns}
\end{frame}

\begin{frame}
  \frametitle{Electromagnetic Kilonova Modeling}
  \begin{itemize}
  \item SuperNu solves Implicit Monte Carlo (IMC) and Discrete
    Diffusion Monte Carlo (DDMC) semi-implicit equations (Fleck \&
    Cumming 1971, Densmore 2007, 2012, Abdikamalov 2012, Cleveland \&
    Gentile 2014).
    \begin{itemize}
    \item Multi-dimensional, and multi-frequency.
    \item Escaping particles are tallied to obtain flux for spectra
      and light curves.
    \end{itemize}
  \item LANL suite of atomic physics codes solves for opacity and
    emissivity ab initio (quantum $\to$ level populations $\to$ opacity).
  \item Research Topics
    \begin{itemize}
    \item Opacity generation and implementation (Fontes et al 2020, 2023)
    \item Source terms (Wollaeger et al 2018, 2019)
    \item Morphology effects (Korobkin et al 2021, Fryer et al 2023)
    \item Active learning + surrogate modeling (Ristic et al 2022)
    \item Data comparison and NLTE (started/underway)
    \end{itemize}
  \end{itemize}
\end{frame}

\begin{frame}
  \frametitle{Electromagnetic Kilonova Modeling}
  \begin{center}
    \includegraphics[width=\columnwidth]{wollaeger-slide-1}
  \end{center}
\end{frame}

\begin{frame}
  \frametitle{Electromagnetic Kilonova Modeling}
  \begin{center}
    \includegraphics[width=\columnwidth]{wollaeger-slide-2}
  \end{center}
\end{frame}

\begin{frame}
  \frametitle{Current/Planned Research Directions}
  \begin{itemize}
  \item Explore zoo of possible sources and engines. How much does the
    merger matter? How diverse is phenomenology? (Student projects:
    Kelsey Lund, Valeria Urrutia-Hurtado)
  \item Gap between end of merger simulations and start of photon
    transport. What happens in the gap? Does it matter? How to fill
    it? (Collaboration with Sanjana Curtis)
  \item Connection to collapsars? What are the sources of r-process in
    the universe? (Student project: Brandon Barker)
  \item Multi-messenger connections? How to best do inferrence?
    (Collaboration with NMMA team. Student, Brendan King)
  \item What can we learn from CCSNe and vice-versa? (Student project:
    Mariam Gogilashvili; collaboration with Carla Frohlich)
  \item Neutrino oscillations are the elephant in the
    room. (Collaboration with Payel Mukhopadhyay, Luke Johns, Gail
    McLaughlin)
  \end{itemize}
\end{frame}

\end{document}