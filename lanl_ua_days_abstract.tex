\documentclass[letter,nofootinbib,superscriptaddress,twocolumn]{revtex4-1}

\newcommand{\nubhlight}{$\nu\texttt{bhlight}$}

\begin{document}

\title{Fusion in Space: Nuclear Astrophysics, Neutron Star Mergers, and Accretion Disks}

\author{Jonah M. Miller}
\email{jonahm@lanl.gov}
\affiliation{Center for Nonlinear Studies, Los Alamos National Laboratory, Los Alamos, NM, USA}
\affiliation{Center for Theoretical Astrophysics, Los Alamos National Laboratory, Los Alamos, NM, USA}
\affiliation{CCS-2, Los Alamos National Laboratory, Los Alamos, NM, USA}

\begin{abstract}
  In the extreme environments achievable in the core of a dying or
  dead star, nuclear reactions---both fusion and fission---are of
  fundamental importance. Fusion of elements lighter than iron and the
  fission of unstable heavy isotopes are both deep reservoirs of
  energy. Moreover, understanding the nuclear reactions occurring in
  space helps us answer questions about our own origins by providing
  insight into the formation of the material of which we are composed.

  In this talk, I will describe one recent, exciting topic in nuclear
  astrophysics---a \textit{kilonova}. In 2017, we observed the merger
  of two neutron stars---ultracompact cores of dead stars. After the
  merger, they probably eventually formed a torus of material orbiting
  around and accreting onto a central black hole. I present
  state-of-the-art numerical models of this ``accretion disk'' system
  and describe how neutron star merger disks contribute to observable
  effects and the total abundance of heavy elements in the universe.
\end{abstract}

\maketitle

\end{document}