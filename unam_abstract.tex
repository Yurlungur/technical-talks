\documentclass[letter,nofootinbib,superscriptaddress,twocolumn]{revtex4-1}

\begin{document}

\title{End-to-End Modeling of a Kilonova}

\author{Jonah M. Miller}
\email{jonahm@lanl.gov}
\affiliation{Center for Theoretical Astrophysics, Los Alamos National Laboratory, Los Alamos, NM, USA}
\affiliation{CCS-2, Los Alamos National Laboratory, Los Alamos, NM, USA}

\begin{abstract}

  The 2017 detection of the in-spiral and merger of two neutron stars
  was a landmark discovery in astrophysics. Through a wealth of
  multi-messenger data, we now know that the merger of these
  ultracompact stellar remnants is a central engine of short gamma ray
  bursts and a site of r-process nucleosynthesis, where the heaviest
  elements in our universe are formed. The radioactive decay of
  unstable heavy elements produced in such mergers powers an optical
  and infra-red transient: The kilonova.

  A key tool in understanding this observation---and the wealth of
  observations we expect going forward---is the forward-modeling of
  the merger and its aftermath. This forward modeling requires many
  inputs, including all four forces of Nature.
  In this talk, I discuss how this modeling is done, from
  end to end: beginning with the in-spiral and ending with electromagnetic
  observations. I also present recent progress in modeling one
  key aspect of the system, the accretion disk formed around the
  compact remnant.

\end{abstract}

\maketitle

\end{document}